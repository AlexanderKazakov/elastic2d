\documentclass[12pt]{article}
\usepackage[utf8]{inputenc}
\usepackage[russian]{babel}
\usepackage{graphicx}
\usepackage[none]{hyphenat}
\usepackage{pgffor}
\usepackage[margin=0.5in]{geometry}

\begin{document}

{\Large \textbf {Отчёт по ЦАГИ -- первое приближение} }

\section{}

При моделировании использованы следующие параметры (г/кубсм, ГПа):

Ортотропный материал: $\rho = 1.6 * m$, упругие модули:

\begin{center}
  \begin{tabular}{| l | l | l | l | l | l | l | l | l |}
    \hline
    c11 & c12 & c13 & c22 & c23 & c33 & c44 & c55 & c66  \\ \hline
    23.3 * m & 6.96 * m & 6.96 * m & 12.0 * m & 6.96 * m & 12.0 * m & 1.67 * m & 5.01 * m & 5.01 * m\\
    \hline
  \end{tabular}
\end{center}

где $ m = 2.5 $ - параметр, позволяющий регулировать акустический импеданс, не меняя скорость звука.
Скорость звука в композите была выбрана согласно отчёту про эксперимент.

Материал призмы (устройство контроля) -- рексолит -- значения взяты с официального сайта:
\begin{eqnarray}
\lambda = 3.03; \\
\mu = 1.39; \\
\rho = 1.05;
\end{eqnarray}

Диаметр призмы согласно отчёту про эксперимент 6.35мм, высота подобрана для соответствия графиков: 8.9мм. Глубина расслоения бралась по данным эксперимента.

Импульс брался как гаусс с шириной 0.1 мкс. Модулирование синусоидой принципиально не меняет вид графиков, поскольку её период порядка ширины гауссианы, а последняя в итоге и так получается как будто модулированная с такой частотой из-за специфики задания граничных условий. Чтобы получить чёткую гауссиану, нужно либо пускать волну сразу в объёме начальным условием, либо изучать конструкцию прибора -- как именно там формируется импульс. 

Дополнительные осцилляции на графиках из моделирования обусловлены боковой отражающей границей  призмы. Возможно, в приборе это как-то демпфируется, поэтому в эксперименте шума меньше.

В пьезокристаллах возникновение электрического поля обусловлено в первую очередь деформацией, а не приложенной силой. Поэтому измеряется, скорее всего, вертикальная компонента скорости, что и построено на графиках для моделирования (точнее, её модуль).

Поскольку устройство датчика в данный момент не известно:
\begin{itemize}
\item геометрия устройства?
\item где расположен кристалл в призме?
\item какое усиление сигнала в какие отрезки времени применяется?
\item есть ли сглаживание или ещё какие-то преобразования сигнала?
\item что значат отдельно все параметры: gain, refgain, contour, distance и т.д?
\item frequency, delay, range, zero?
\item длина линии задержки и другие параметры в начале документа?
\item velocity 2733 -- это скорость звука в композите вдоль направления контроля? выставляется самостоятельно?
\item каков реальный контакт датчика с материалом,
\end{itemize}
указанные параметры вполне могут сильно куда-то сместится, так как здесь легко свалиться в локальный минимум.


\section{}

На графиках из расчёта по оси времени указаны микросекунды, на графиках из эксперимента -- половинное расстояние, которое волна проходит за время в композите.
Стробы G1 и G2 -- это промежутки времени, в которые датчик ожидает появление сигнала, превышающего пороговую амплитуду. Когда и если сигнал пришёл, время его начала фиксируется как время отклика.

Учитывая вышесказанное, графики из эксперимента №1 и №4 не верны: расстояние по оси абсцисс между временами фиксации сигналов G1 и G2 не соответствуют указанному снизу значению. Этим и обусловлено несовпадение с моделированием этих случаев. Можно сравнить с очень близкими случаями №2 и №3 -- там всё верно и совпадает.


\newpage

\foreach \n in {0,...,16} {

	\begin{figure}[!htb]
	  \centerline{\includegraphics[width=0.8\linewidth]{../sim/png/\n.png}}
	  \caption{Моделирование файл \n } \label{fig:sim\n}
	\end{figure}
	\begin{figure}[!htb]
	  \centerline{\includegraphics[width=0.7\linewidth]{../exp/\n.png}}
	  \caption{Эксперимент файл \n }\label{fig:exp\n}
	\end{figure}

}

\end{document}
